\documentclass{article}

% Language setting
% Replace `english' with e.g. `spanish' to change the document language
\usepackage[english]{babel}

% Set page size and margins
% Replace `letterpaper' with `a4paper' for UK/EU standard size
\usepackage[letterpaper,top=2cm,bottom=2cm,left=3cm,right=3cm,marginparwidth=1.75cm]{geometry}

% Useful packages
\usepackage{amsmath}
\usepackage{graphicx}
\usepackage[colorlinks=true, allcolors=blue]{hyperref}

\title{PS12}
\author{Anais Ouedraogo}

\begin{document}
\maketitle


\begin{itemize}
\item The mean value of exper is almost half of the mean value of hgc. The median value of exper is exactly half of the median value of hgc.
\item lowage is missing at a rate of 31\%. I think that it might be MAR
\end{itemize}


\begin{table}
\centering
\begin{tabular}[t]{lrrrrrrr}
\toprule
  & Unique (\#) & Missing (\%) & Mean & SD & Min & Median & Max\\
\midrule
logwage & 1546 & 31 & \num{1.7} & \num{0.7} & \num{-1.0} & \num{1.7} & \num{4.2}\\
hgc & 14 & 0 & \num{12.5} & \num{2.4} & \num{5.0} & \num{12.0} & \num{18.0}\\
exper & 1932 & 0 & \num{6.4} & \num{4.9} & \num{0.0} & \num{6.0} & \num{25.0}\\
kids & 2 & 0 & \num{0.4} & \num{0.5} & \num{0.0} & \num{0.0} & \num{1.0}\\
\bottomrule
\end{tabular}
\end{table}

\begin{itemize}
\item The value of b1​ using the listwise deletion and mean imputation is the same: 0.059. The value of b1​ using the Heckman method is −1.104. There is a 0.032 difference between the true value of b1​ and b1​ in the first two methods, while there is a 1.195 difference between the true value of b1​ and b1​ in the last method. I can conclude that mean imputation and listwise deletion seem to be more accurate in this case compared to the Heckman method.
\end{itemize}

\begin{table}
\centering
\begin{tabular}[t]{lccc}
\toprule
  & (Listwise Deletion) & (Mean Imputation) & (Heckman Selection)\\
\midrule
(Intercept) & \num{0.834} & \num{0.834} & \num{0.446}\\
 & \num{0.834} & \num{0.834} & \num{20.553}\\
 & (\num{0.113}) & (\num{0.113}) & (\num{0.122})\\
 & (\num{0.113}) & (\num{0.113}) & (\num{1.111})\\
hgc & \num{0.059} & \num{0.059} & \num{-1.104}\\
 & \num{0.059} & \num{0.059} & \num{0.091}\\
 & (\num{0.009}) & (\num{0.009}) & (\num{0.010})\\
 & (\num{0.009}) & (\num{0.009}) & (\num{0.066})\\
union1 & \num{0.222} & \num{0.222} & \num{-1.113}\\
 & \num{0.222} & \num{0.222} & \num{0.186}\\
 & (\num{0.087}) & (\num{0.087}) & (\num{0.084})\\
 & (\num{0.087}) & (\num{0.087}) & (\num{0.213})\\
college1 & \num{-0.065} & \num{-0.065} & \num{-0.565}\\
 & \num{-0.065} & \num{-0.065} & \num{0.092}\\
 & (\num{0.106}) & (\num{0.106}) & (\num{0.100})\\
 & (\num{0.106}) & (\num{0.106}) & (\num{0.227})\\
exper & \num{0.050} & \num{0.050} & \num{-0.506}\\
 & \num{0.050} & \num{0.050} & \num{0.054}\\
 & (\num{0.013}) & (\num{0.013}) & (\num{0.012})\\
 & (\num{0.013}) & (\num{0.013}) & (\num{0.030})\\
 & \num{-0.004} & \num{-0.004} & \num{-0.002}\\
 & (\num{0.001}) & (\num{0.001}) & (\num{0.001})\\
married1 &  &  & \num{-2.275}\\
 &  &  & (\num{0.162})\\
kids &  &  & \num{0.495}\\
 &  &  & (\num{0.114})\\
invMillsRatio &  &  & \num{-0.695}\\
 &  &  & (\num{0.060})\\
sigma &  &  & \num{0.696}\\
rho &  &  & \num{-0.998}\\
\midrule
Num.Obs. & \num{1545} & \num{1545} & \num{2229}\\
R2 & \num{0.038} & \num{0.038} & \num{0.092}\\
R2 Adj. & \num{0.035} & \num{0.035} & \num{0.088}\\
AIC & \num{3182.4} & \num{3182.4} & \\
BIC & \num{3219.8} & \num{3219.8} & \\
Log.Lik. & \num{-1584.189} & \num{-1584.189} & \\
F & \num{12.106} & \num{12.106} & \\
RMSE & \num{0.67} & \num{0.67} & \num{0.66}\\
\bottomrule
\end{tabular}
\end{table}




\begin{itemize}
\item Probit model: The counterfactual policy did not have a significant effect the mean probability changed from 0.2373 to 0.2278. I think that the model estimated above is not realistic because the probability should be higher and a counterfactual policy should have had a bigger effect.

\end{itemize}


\end{document}